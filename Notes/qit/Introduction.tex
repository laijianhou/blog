

\section{Introduction}
\begin{Idea}{Quantum Bit}{}
    A quantum bit (qubit) is a superposition (linear combination) of two basis states, $\ket{0}$ and $\ket{1}$:
    \begin{equation}
        \ket{\Psi} = \psi_0\ket{0} + \psi_1\ket{1}\,,
    \end{equation}
    where 
    \begin{equation}
        \ket{0}= \begin{pmatrix}
            1 \\
            0
            \end{pmatrix}, \quad \ket{1} = \begin{pmatrix}
                0 \\
                1
            \end{pmatrix}\,,
    \end{equation}
    $\psi_0$ and $\psi_1$ are complex numbers called \textbf{probability amplitudes}.

\end{Idea}

\begin{Idea}{Postulates in Quantum Mechanics}{}
    \hl{testing}
\end{Idea}

\begin{Idea}{Outer Product}{}
    Suppoese that $\ket{\Psi}$ and $\ket{\Phi}$ are two state vectors, then the \textbf{outer product} between $\ket{\Psi}$ and $\ket{\Phi}$ is defined as
    \begin{equation}
        \ketbra{\Psi}{\Phi} = \ket{\Psi}\otimes \bra{\Phi}\,.
    \end{equation}
    It bahaves like a matrix multiplication if $\ket{\Psi}$ is a $m\times 1$ column vector while $\bra{\Phi}$ is a $1\times n$ row vector, then the resulting outer product $\ketbra{\Psi}{\Phi}$ is a $m\times n$ matrix.

\end{Idea}

\begin{Idea}{Hermitian Operator}{}
    An operator $A$ is said to be \textbf{Hermitian} if it satisfies 
    \begin{equation}
        A^\dagger = A\,,
    \end{equation}
    where $\dagger$ means \textbf{Hermitian adjoint} or \textbf{conjugate transpose}.
\end{Idea}


\begin{Idea}{Unitary Transformation}{block_Uni.Trans.}
    An unitary transformation (operator) $U$ satisfies 
    \begin{equation}
    UU^\dagger=\mathbb{1} \implies U^{-1}=U^\dagger\,.
    \end{equation}
\end{Idea}

In contrast with the normal quantum mechanics which the states was described by complex state vector, in quantum information, we use \textbf{density matrix} to describe our states. \hl{Density matrices formalism}
\begin{Idea}{Density Operator/ Density Matrix}{block_density.matrix}
    \hl{Sort of repeated with the ``pure states'' idea}  Suppose that $X$ is a system and $\Sigma$ is its \hl{classical state set}. \textbf{Density matrix}, $\rho$ describing state $X$ is a matrix\footnote{A linear mapping.} with complex number entries whose rows and columns have been placed in correspondence with $\Sigma$.

    Properties of density matrices:
    \begin{enumerate}
        \item $\Tr(\rho) = 1$.
        \item $\rho$ is positive semidefinite: $\rho \geq 0$\footnote{If you diagonalise this matrix, the eigenvalues will non-negative. It implies $\rho$ is Hermitian \hl{?}}.
        \item Diagonal entries are the probabilities for each classical state to appear from a standard \hl{basis measurement}. 
        \item Off-diagonal entries describes \hl{how} the two corresponding classical states are in quantum superposition. (coherences/ how pure or how mixed of states)
    \end{enumerate}
\end{Idea}



Here are some motivation with this formalism for QI:
\begin{itemize}
    \item Able to represent broader class of quantum states, like quantum states with uncertainty or randomness.
    \item Can describe states of isolated parts of the systems, like one system that's in entangled with another system.

    \item Able to describe classical and quantum information together within single mathematical framework.
\end{itemize}


\begin{Idea}{Pure States}{block_Pure.States}
    Suppose we have a single, isolated quantum state vector $\ket{\Psi}$, which is a column vector having Euclidean norm 1. With Spectral Theorem (\Cref{thm:block_spect.thrm.}), the state vector can be written as the linear combination of the basis:
    \begin{equation}
        \ket{\Psi} = \sum_{i=1}^n\psi_i\ket{i} = \sum_{i=0}^{n-1}\psi_i\ket{i}\,.
    \end{equation}
    The form in the second equality is normally adapted in QIT so it can start with  computational basis state, $\ket{0}$.
    
    Then, the density matrix representation of the same state is called \textbf{pure state} (Notice that pure state is a projector on the state $\ket{\Psi}$):
    \begin{equation}\label{eqn:rho}
        \rho = \ketbra{\Psi}{\Psi}\,.
    \end{equation}
    $\rho$ has these properties:
    \begin{enumerate}
        \item $\rho$ is Hermitian ($\rho^\dagger=\rho)$.
        \item $\Tr(\rho) = \Tr(\rho^2)=1$.
    \end{enumerate}
\end{Idea}

\begin{Proof}{Properties in \Cref{idea:block_Pure.States}}{}
    \begin{enumerate}
        \item To start with, we take Hermitian on \Cref{eqn:rho}:
            \begin{align*}
                \rho^{\dagger} &= (\ketbra{\Psi}{\Psi})^\dagger \\
                &= \bra{\Psi}^\dagger\ket{\Psi}^\dagger \\
                &= \left(\sum_{i=1}^n \psi_i^*\bra{i} \right)^\dagger \left(\sum_{j=1}^n\psi_j\ket{j}\right)^\dagger \\
                &= \left(\sum_{i=1}^n\psi_i\ket{i} \right)\left(\sum_{j=1}^n\psi_j^*\bra{j}\right)
                \\
                &= \ketbra{\Psi}{\Psi} = \rho\,.
            \end{align*}
            Because of $\rho^\dagger=\rho$, we conclude that $\rho$ is Hermitian. Intuitively, we can think of the ket is a column matrix, and bra is a row matrix, hence $\ket{\Psi}^\dagger=\bra{\Psi}$.$\qed$ 

        \item \Cref{eqn:rho} can be written as
        \begin{align*}
            \rho = \ketbra{\Psi}{\Psi} &= \sum_{i,j=1}^n \psi_i\psi_j^*\ketbra{i}{j} \\
            &= \begin{pmatrix}
                |\psi_1|^2 & \psi_1\psi_2^* & \dots & \psi_1\psi_n^* \\
                \psi_2\psi_1^* & |\psi_2|^2 & \dots & \psi_2\psi_n^* \\
                \vdots & \vdots & \ddots & \vdots \\
                \psi_n\psi_1^* & \psi_n\psi_2^* & \dots & |\psi_n|^2
            \end{pmatrix}
        \end{align*}
        Hence,
        \begin{equation}
            \Tr(\rho) = \sum_{i=1}^n|\psi_i|^2=1\,,
        \end{equation}
        which is equal to total probability, by definition. Next,
        \begin{equation}
            \Tr(\rho^2) = \Tr(\ket{\Psi}\braket{\Psi|\Psi}\bra{\Psi}) = \Tr(\ketbra{\Psi}{\Psi}) = \Tr(\rho) = 1\,,
        \end{equation}
        where we used the inner product of itself is equal to 1 because the state is normalised. $\qed$
    \end{enumerate}
\end{Proof}

\begin{Idea}{Multiqubit States}{}
For $n$ qubits, we can represent the state as 
\begin{equation}
    \ket{\Psi} = \sum_{i=0}^N \psi_i\ket{i} =  \begin{pmatrix}
        \psi_0 \\
        \psi_1 \\
        \vdots \\
        \psi_N
    \end{pmatrix}\,,
\end{equation}
where $N=2^n-1$. Then, the density matrix of $n$ qubits is
        \begin{align*}
            \rho = \ketbra{\Psi}{\Psi} &= \sum_{i,j=0}^n \psi_i\psi_j^*\ketbra{i}{j} \\
            &= \begin{pmatrix}
                |\psi_0|^2 & \psi_0\psi_1^* & \dots & \psi_0\psi_N^* \\
                \psi_1\psi_0^* & |\psi_1|^2 & \dots & \psi_1\psi_N^* \\
                \vdots & \vdots & \ddots & \vdots \\
                \psi_N\psi_1^* & \psi_N\psi_2^* & \dots & |\psi_N|^2
            \end{pmatrix}\,.
        \end{align*}        


    \begin{equation}
        \ket{0}\otimes \ket{0} = \begin{pmatrix}
            1 \\
            0
        \end{pmatrix} \otimes \begin{pmatrix}
            1 \\
            0
        \end{pmatrix} = \begin{pmatrix}
            1 \\
            0 \\
            0 \\
            0
        \end{pmatrix}
    \end{equation}
    \hl{???}
\end{Idea}

\begin{Idea}{Equation of Motion}{block_EOM}
    $\rho$ satisfies equation of motion, or known as Liouville-von Neumann equation:
\end{Idea}

\hl{no global phase degeneracy}

\begin{Idea}{Mixed States}{block_mixed}
    Mixed states is an ensemble of pure states which associated with classical probability of occurrence. Let $\{\rho_i\}_{i=1}^n = \{\rho_1,\rho_2, \dots , \rho_{n}\}$ be the set of density matrices of the states, and  $\{p_i\}_{i=1}^n=\{p_1,p_2,\dots , p_{n})$ be the set of probabilities. Suppose we prepare a system in state $\rho_k$ with probability $p_k$,
    then the mixed state is represented by this density matrix,
    \begin{equation}
        \rho  = \sum_{i=1}^{n} p_i\rho_i = \sum_{i=1}^np_i\ketbra{\Psi_i}{\Psi_i}\,.
    \end{equation}
    It's also known as the convex combination of density matrices. The density matrix for mixed state has these properties:
\begin{enumerate}
    \item $\Tr(\rho)=1$.
    \item $\Tr(\rho^2)<1$.
\end{enumerate}
\end{Idea}
Density matrices don't describe how the system was prepared, but it tells you how system changed after some action performed. 

\hl{convex combination in maths}

\begin{Proof}{Properties of \Cref{idea:block_mixed}}{}
    Let $\rho$ be the density matrix of the mixed state:
    \begin{equation}
        \rho = \sum_{i=1}^np_i\rho_i\,,
    \end{equation}
    where
    \begin{equation}
        \rho_i = \sum_{j=1}^m\sum_{k=1}^m\psi_j^{(i)}\psi_k^{(i)*}\ketbra{j}{k}\,.
    \end{equation}
    \begin{enumerate}
        \item Then, 
        \begin{equation}
            \Tr(\rho) = 
        \end{equation}
    \end{enumerate}
    
\end{Proof}



\begin{Idea}{Expectation Values}{block_Exp.values}
    \begin{align*}
        \braket{A} &= \braket{\Psi|\hat{A}|\Psi} \\
                   &= \sum_{i=1}^n \braket{\Psi|\hat{A}|i}\braket{i|\Psi} \\ 
                   &= \sum_{i=1}^n \braket{i|\Psi}\braket{\Psi|\hat{A}|i} \\
                   &= \sum_{i=1}^n \braket{i|\rho\hat{A}|i} \\
                   &= \Tr(\rho\hat{A})\,.
    \end{align*}    
    So, we can interpret it as finding the trace of the matrix multiplication between density matrix and matrix (operator) A.
\end{Idea}



\begin{Idea}{Probabilistic States}{block_prob.states}

\end{Idea}

\begin{Idea}{Probabilistic Measurement}{block_prob.meas.}
    
\end{Idea}


\hl{normal matrices}


\begin{Theorem}{Spectral Theorem}{block_spect.thrm.}
    
\end{Theorem}

\begin{Theorem}{Spectral Theorem for Positive Semidefinite Matrices}{block_spec.thrm.semipos}
    
\end{Theorem}

\begin{Idea}{Bloch Sphere/ Poincar\'{e} Sphere}{block_Bloch}
    
\end{Idea}


\begin{Idea}{Bloch Ball}{block_BlochBall}
    
\end{Idea}
\hl{gemetrical representation beyond single qubit}

\begin{Idea}{Pauli Matrices}{block_Pauli}
    Pauli matrices is defined as 
    \begin{equation}
        \sigma_x = 
        \begin{pmatrix}
            0 & 1 \\
            1 & 0
        \end{pmatrix}, \quad \sigma_y = \begin{pmatrix}
            0 & -i \\
            i & 0
        \end{pmatrix}, \quad \sigma_z = \begin{pmatrix}
            1 & 0 \\
            0 & -1
        \end{pmatrix}\,.
    \end{equation}
    More compactly, we can write Pauli matrices as $\sigma_j$, where $j=1,2,3$, corresponds to $x,y,z$ respectively. Pauli matrices have some properties:
    \begin{enumerate}
        \item The anti-commutator of Pauli matrices satisfies $\{\sigma_j, \sigma_k\}=2\delta_{jk}\mathbb{1}$.

        \item The commutator (commutation relation) of Pauli matrices satisfies $[\sigma_j,\sigma_k]=2i\epsilon_{jkl}\sigma_l$. 

        \item Pauli matrices are traceless matrices. i.e., $\Tr(\sigma_j)=0.$

        \item Pauli matrices are self-inverse matrices. i.e., $\sigma_j^{-1}=\sigma_j$.
    \end{enumerate}
\end{Idea}

\begin{Proof}{Properties of \Cref{idea:block_Pauli}}{}
Notice that
            \begin{align}
                            \sigma_x\sigma_y &= \paulix\pauliy = \begin{pmatrix}
                i & 0 \\
                0 & -i
            \end{pmatrix}\,, \label{eqns:PauliMatricesXY}\\
            \sigma_y\sigma_x &=  \pauliy\paulix =\begin{pmatrix}
                -i & 0 \\
                0 & i
            \end{pmatrix}\,,  \label{eqns:PauliMatricesYX}\\
            \sigma_y\sigma_z &= \pauliy\pauliz = \begin{pmatrix}
                0 & i \\ 
                i & 0
            \end{pmatrix}\,, \label{eqns:PauliMatricesYZ}\\
            \sigma_z\sigma_y &= \pauliz\pauliy = \begin{pmatrix}
                0 & -i \\
                -i & 0
            \end{pmatrix}\,, \label{eqns:PauliMatricesZY}\\
            \sigma_x\sigma_z &= \paulix \pauliz = \begin{pmatrix}
                0 & -1 \\
                1 & 0
            \end{pmatrix}\,, \label{eqns:PauliMatricesXZ}\\
            \sigma_z\sigma_x &= \pauliz\paulix = \begin{pmatrix}
                0 & 1 \\
                -1 & 0
            \end{pmatrix}\,, \label{eqns:PauliMatricesZX}\\
            \sigma_x\sigma_x &= \paulix\paulix = \begin{pmatrix}
                1 & 0\\
                0 & 1
            \end{pmatrix} = \mathbb{1}\,, \label{eqns:PauliMatricesXX}\\
            \sigma_y\sigma_y &= \pauliy\pauliy = \begin{pmatrix}
                1 & 0\\
                0 & 1
            \end{pmatrix} = \mathbb{1}\,, \label{eqns:PauliMatricesYY}\\
            \sigma_z\sigma_z &= \pauliz\pauliz = \begin{pmatrix}
                1 & 0\\
                0 & 1
            \end{pmatrix} = \mathbb{1} \,. \label{eqns:PauliMatricesZZ}
            \end{align}
            
    \begin{enumerate}
    \item Hence, from equations above, we will get 
            \begin{align}
                \sigma_x\sigma_y + \sigma_y\sigma_x &= 0 \Rightarrow \{\sigma_x,\sigma_y\} = 0 \,, \\
                \sigma_x\sigma_z + \sigma_z\sigma_x &= 0 \Rightarrow \{\sigma_x,\sigma_z\} = 0 \,,\\
                \sigma_y\sigma_z + \sigma_z\sigma_y &= 0 \Rightarrow \{\sigma_y,\sigma_z\} = 0\,,
            \end{align}
            and 
            \begin{align}
                \sigma_x\sigma_x + \sigma_x\sigma_x &= 2\mathbb{1} \,,\\
                \sigma_y\sigma_y + \sigma_y\sigma_y &= 2\mathbb{1} \,,\\
                \sigma_z\sigma_z + \sigma_z\sigma_z &= 2\mathbb{1}\,.
            \end{align}
            Thus, we can summarise it as
            \begin{equation}
                \{\sigma_j,\sigma_k \} = 2\delta_{jk}\mathbb{1}\,. \qed 
            \end{equation}

    \item From \Cref{eqns:PauliMatricesXX,eqns:PauliMatricesXZ,eqns:PauliMatricesXY,eqns:PauliMatricesYX,eqns:PauliMatricesYY,eqns:PauliMatricesYZ,eqns:PauliMatricesZX,eqns:PauliMatricesZY,eqns:PauliMatricesZZ}, we will get
    \begin{align}
        [\sigma_x,\sigma_y] &= \sigma_x\sigma_y - \sigma_y\sigma_x = 2i\begin{pmatrix}
            1 & 0 \\
            0 & -1
        \end{pmatrix} = 2i\sigma_z \,,
        \\
        [\sigma_y,\sigma_z] &= \sigma_y\sigma_z - \sigma_z\sigma_y = 2i\begin{pmatrix}
          0 & 1 \\
          1 & 0
        \end{pmatrix} = 2i\sigma_x \,,
        \\
        [\sigma_z,\sigma_x] &= \sigma_z\sigma_x - \sigma_x\sigma_z = 2i\begin{pmatrix}
            0 & -i \\
            i & 0
        \end{pmatrix} = 2i\sigma_y \,,
    \end{align}
    Hence, 
    \begin{align}
        [\sigma_y,\sigma_x] &= -[\sigma_x,\sigma_y] = -2i\sigma_z\,, \\
        [\sigma_z,\sigma_y] &= -[\sigma_y,\sigma_z] = -2i\sigma_x\,, \\
        [\sigma_x,\sigma_z] &= -[\sigma_z,\sigma_x] = -2i\sigma_y\,.
    \end{align}
    With the equations above, and the fact that Pauli matrices (operators in quantum mechanics) are always commute with itself, we conclude that 
    \begin{equation}
        [\sigma_j,\sigma_k] = 2i\epsilon_{jkl}\sigma_l \qed
    \end{equation}

    \item Obvious. $\qed$

    \item Obvious from \Cref{eqns:PauliMatricesXX,eqns:PauliMatricesYY,eqns:PauliMatricesZZ}.$\qed$

    \end{enumerate}
\end{Proof}
Notice that Pauli matrices satisfies the equation 
    \begin{equation}
        \sigma_j\sigma_k = \delta_{jk}\mathbb{1}+i\epsilon_{jkl}\sigma_l\,.
    \end{equation}



\begin{Idea}{Reduced States and Partial Trace for Bipartite Systems}{block_ReducedStates}
Suppose systems $A$ and $B$ are in a joint pure state $\ket{\Psi}$, so the composite density matrix is $\rho_{AB}=\ketbra{\Psi}{\Psi}$. Then, the reduced state for $A$ is obtained by tracing over $B$ and is given by
\begin{equation}\label{eqn:reduced_A}
    \rho_A = \Tr_B(\rho_{AB})=\sum_{b}(\mathbb{1}_A\otimes\bra{b})\rho_{AB}(\mathbb{1}_A \otimes \ket{b})\,,
\end{equation}
where $\{\ket{b}\}$ is an orthonormal basis for system $B$. Similarly, the reduced state for $B$ is:
\begin{equation}\label{eqn:reduced_B}
    \rho_B = \Tr_A(\rho_{AB}) =  \sum_{a}(\bra{a}\otimes\mathbb{1}_B)\rho_{AB}(\ket{a}\otimes \mathbb{1}_B)\,,
\end{equation}
where $\{\ket{a}\}$ is an orthonormal basis for system $A$. Note that, for example, $\Tr_A(\rho)$ means taking partial trace of $\rho$.

\hl{prove this}
\end{Idea}
For example, suppose we have a joint two-qubit pure state, generally (not necessary to be entangled) described as $\ket{\Psi}$, then 
\begin{equation}
    \ket{\Psi} = \sum_{i=0}^{1}\sum_{j=0}^1 \psi_{ij}\ket{ij} = \psi_{00}\ket{00}+\psi_{01}\ket{01}+\psi_{10}\ket{10} + \psi_{11}\ket{11}\,.
\end{equation}
Note that: Two qubits is said to be entangled to each other iff the state can be expressed as $\ket{\Psi} = \ket{\Phi_A}\otimes\ket{\Phi_B}$. i.e., $\psi_{ij}$ can be written as $\psi_{ij}=\phi_i\phi_{j}$, so \begin{equation}
    \ket{\Psi} = \sum_{ij}\psi_{ij}\ket{ij}=\left(\sum_i\phi_i \ket{i}\right)\otimes \left(\sum_j\phi_j \ket{j}\right) = \ket{\Phi_A}\otimes\ket{\Phi_B}\,.
\end{equation}
Then, we can prepare the density matrix for this joint state as
\begin{align*}
    \rho_{AB} &= \ketbra{\Psi}{\Psi} \\
              &= \begin{pmatrix}
\psi_{00} \\
\psi_{01} \\
\psi_{10} \\
\psi_{11}
\end{pmatrix}
\begin{pmatrix}
\psi_{00}^* & \psi_{01}^* & \psi_{10}^* & \psi_{11}^*
\end{pmatrix} \\
 \rho_{AB}   &= \begin{pmatrix}
|\psi_{00}|^2 & \psi_{00}\psi_{01}^* & \psi_{00}\psi_{10}^* & \psi_{00}\psi_{11}^* \\
\psi_{01}\psi_{00}^* & |\psi_{01}|^2 & \psi_{01}\psi_{10}^* & \psi_{01}\psi_{11}^* \\
\psi_{10}\psi_{00}^* & \psi_{10}\psi_{01}^* & |\psi_{10}|^2 & \psi_{10}\psi_{11}^* \\
\psi_{11}\psi_{00}^* & \psi_{11}\psi_{01}^* & \psi_{11}\psi_{10}^* & |\psi_{11}|^2
\end{pmatrix}\,. \numberthis\label{eqn:rho_AB}
\end{align*}
Or, generally,
\begin{equation}\label{eqn:rho_AB_tensor}
    \rho_{AB} = \sum_{ij}\psi_{ij}\ket{ij}\left(\sum_{jk}\psi_{kl}^*\bra{kl}\right) = \sum_{ijkl}\psi_{ij}\psi^*_{kl}\ketbra{ij}{kl} = \sum_{ijkl}\psi_{ij}\psi^*_{kl}\left(\ketbra{i}{k}\otimes\ketbra{j}{l}\right)
\end{equation}
Then, using \Cref{eqn:reduced_A} and \Cref{eqn:rho_AB_tensor}, we can get the reduced state for system $A$ as
\begin{align*}
    \rho_A = \Tr_B(\rho_{AB)} &= \sum_b(\mathbb{1}_A\otimes \bra{b})\rho_{AB}(\mathbb{1}_A\otimes\ket{b}) \\
    &= \sum_b(\mathbb{1}_A\otimes\bra{b})\left(\sum_{ijkl}\psi_{ij}\psi^*_{kl}\ketbra{i}{k}\otimes\ketbra{j}{l} \right)(\mathbb{1}_A\otimes\ket{b}) \\
    &= \sum_{bijkl}\psi_{ij}\psi_{kl}^*(\ketbra{i}{k})(\braket{b|j}\braket{l|b}) \\
    &= \sum_{bijkl}\psi_{ij}\psi_{kl}^*(\ketbra{i}{k})\delta_{bj}\delta_{lb} \\
    &= \sum_{bik}\psi_{ib}\psi_{kb}^*\ketbra{i}{k} \\
    &= \sum_{ik}\left(\psi_{i0}\psi_{k0}^*\ketbra{i}{k} + \psi_{i1}\psi_{k1}^* \ketbra{i}{k} \right) \\
    &= (|\psi_{00}|^2 |\psi_{01}|^2)\ketbra{0}{0} +(\psi_{00}\psi_{10}^* + \psi_{01}\psi_11^*)\ketbra{0}{1} +(\psi_{10}\psi^*_{00} + \psi_{11}\psi_{01}^*)\ketbra{1}{0} +(|\psi_{10}|^2 + |\psi_{11}|^2 )\ketbra{1}{1} \\
    &= \begin{pmatrix}
|\psi_{00}|^2 + |\psi_{01}|^2 & \psi_{00}\psi_{10}^* + \psi_{01}\psi_{11}^* \\
\psi_{10}\psi_{00}^* + \psi_{11}\psi_{01}^* & |\psi_{10}|^2 + |\psi_{11}|^2
\end{pmatrix}
\end{align*}
Compare to \Cref{eqn:rho_AB}, 
\begin{equation}
     \rho_{AB}   = \begin{pNiceMatrix}[margin, cell-space-limits=3pt]
\Block[fill=red!15]{2-2}{} |\psi_{00}|^2 & \psi_{00}\psi_{01}^* & \Block[fill=yellow!50]{2-2}{}  \psi_{00}\psi_{10}^* & \psi_{00}\psi_{11}^* \\
\psi_{01}\psi_{00}^* & |\psi_{01}|^2 & \psi_{01}\psi_{10}^* & \psi_{01}\psi_{11}^* \\
\Block[fill=green!15]{2-2}{} \psi_{10}\psi_{00}^* & \psi_{10}\psi_{01}^* & \Block[fill=blue!15]{2-2}{}  |\psi_{10}|^2 & \psi_{10}\psi_{11}^* \\
\psi_{11}\psi_{00}^* & \psi_{11}\psi_{01}^* & \psi_{11}\psi_{10}^* & |\psi_{11}|^2
\end{pNiceMatrix}\,,
\end{equation}
the partial trace over $B$ simply can be done by
\begin{equation}
    \Tr_B(\rho_{AB}) = \begin{pmatrix}
\Tr\begin{pNiceMatrix}[margin, cell-space-limits=3pt]
\Block[fill=red!15]{2-2}{}|\psi_{00}|^2 & \psi_{00}\psi_{01}^* \\
\psi_{01}\psi_{00}^* & |\psi_{01}|^2
\end{pNiceMatrix}
&
\Tr\begin{pNiceMatrix}[margin, cell-space-limits=3pt]
\Block[fill=yellow!50]{2-2}{} \psi_{00}\psi_{10}^* & \psi_{00}\psi_{11}^* \\
\psi_{01}\psi_{10}^* & \psi_{01}\psi_{11}^*
\end{pNiceMatrix}
\\[1.2em]
\Tr\begin{pNiceMatrix}[margin, cell-space-limits=3pt]
\Block[fill=green!15]{2-2}{}\psi_{10}\psi_{00}^* & \psi_{10}\psi_{01}^* \\
\psi_{11}\psi_{00}^* & \psi_{11}\psi_{01}^*
\end{pNiceMatrix}
&
\Tr\begin{pNiceMatrix}[margin, cell-space-limits=3pt]
\Block[fill=blue!15]{2-2}{}|\psi_{10}|^2 & \psi_{10}\psi_{11}^* \\
\psi_{11}\psi_{10}^* & |\psi_{11}|^2
\end{pNiceMatrix}
\end{pmatrix}\,.
\end{equation}
Similarly using \Cref{eqn:reduced_B}, we can get the reduced state for system $B$ as
\begin{align*}
\rho_B = \Tr_A(\rho_{AB})
&=
\sum_a
\bigl(
\bra{a}\otimes \mathbb{1}_B
\bigr)
\rho_{AB}
\bigl(
\ket{a}\otimes \mathbb{1}_B
\bigr)
\\
&=
\sum_{aijkl}
\psi_{ij}\psi_{kl}^*
\bigl(
\bra{a}\otimes \mathbb{1}_B
\bigr)
\bigl(
\ket{i}\bra{k}\otimes \ketbra{j}{l}
\bigr)
\bigl(
\ket{a}\otimes \mathbb{1}_B
\bigr)
\\
&=
\sum_{aijkl}
\psi_{ij}\psi_{kl}^*
\,\braket{a|i}\braket{k|a}\,
\ketbra{j}{l}
\\[0.6em]
&=
\sum_{aijkl}
\psi_{ij}\psi_{kl}^*
\,\delta_{ai}\delta_{ka}\,
\ketbra{j}{l}
\\
&=
\sum_{ajl}
\psi_{aj}\psi_{al}^*
\ketbra{j}{l}
\label{eq:rhoB-general}
\\
&=
\sum_{jl}
\psi_{0j}\psi_{0l}^*\ketbra{j}{l}
+
\sum_{jl}
\psi_{1j}\psi_{1l}^*\ketbra{j}{l} 
\\
&= \begin{pmatrix}
|\psi_{00}|^2 + |\psi_{10}|^2
&
\psi_{00}\psi_{01}^* + \psi_{10}\psi_{11}^*
\\[0.6em]
\psi_{01}\psi_{00}^* + \psi_{11}\psi_{10}^*
&
|\psi_{01}|^2 + |\psi_{11}|^2
\end{pmatrix}\,.
\end{align*}
Again, compare to \Cref{eqn:rho_AB}, 
\begin{equation}
    \rho_{AB} = 
    \begin{pNiceMatrix}[margin, cell-space-limits=3pt]
    % Top-left 2x2 block
    \Block[fill=red!15]{2-2}{} 
    |\psi_{00}|^2 & \psi_{00}\psi_{01}^* & \psi_{00}\psi_{10}^* & \psi_{00}\psi_{11}^* \\
    \psi_{01}\psi_{00}^* & |\psi_{01}|^2 & \psi_{01}\psi_{10}^* & \psi_{01}\psi_{11}^* \\
    % Bottom-right 2x2 block
    \psi_{10}\psi_{00}^* & \psi_{10}\psi_{01}^* & \Block[fill=green!15]{2-2}{} 
    |\psi_{10}|^2 & \psi_{10}\psi_{11}^* \\
    \psi_{11}\psi_{00}^* & \psi_{11}\psi_{01}^* & \psi_{11}\psi_{10}^* & |\psi_{11}|^2
    \end{pNiceMatrix}
\end{equation}

the partial trace over $A$ simply can be done by
\begin{equation} 
\Tr_A(\rho_{AB}) = 
\begin{pNiceMatrix}[margin, cell-space-limits=3pt]
    \Block[fill=red!15]{2-2}{} 
    |\psi_{00}|^2 & \psi_{00}\psi_{01}^*  \\
    \psi_{01}\psi_{00}^* & |\psi_{01}|^2 
\end{pNiceMatrix} + 
\begin{pNiceMatrix}[margin, cell-space-limits=3pt]
    \Block[fill=green!15]{2-2}{} 
    |\psi_{10}|^2 & \psi_{10}\psi_{11}^* \\
    \psi_{11}\psi_{10}^* & |\psi_{11}|^2
\end{pNiceMatrix}
\end{equation}\,.



\begin{Idea}{}{block_par.traces}
\hl{Pending}

    \textit{Equivalent statement:} $\Tr_A$ and $\Tr_B$ are the unique linear mappings for which these equations are always true:
    \begin{align}
        \Tr_A(\mathcal{M}\otimes\mathcal{N}) &=\Tr(\mathcal{M})\mathcal{N} \\
        \Tr_B(\mathcal{M}\otimes\mathcal{N}) &= \Tr(\mathcal{N})\mathcal{M}
    \end{align}
\end{Idea}




\begin{itemize}
    \item \hl{Pauli errors}
    \item \hl{Cluster states}
    \item \hl{Bures length}
    \item \hl{Weighted distances in quantum information (e.g. Weighted Bures length)}
    \item \hl{Weighted Hilbert-Schmidt distance}
    \item \hl{Pauli rotation}
\end{itemize}


\begin{Idea}{Evolution of State Through Unitary Operators}{}
    For example, Hadamard state:
    \begin{equation}
        H\ket{0} = \ket{+} = \frac{1}{\sqrt{2}}(\ket{0}+\ket{1})
    \end{equation}
    \hl{continue}
\end{Idea}

\begin{Idea}{Stabilizer Formalism}{}
    \hl{Pending}
\end{Idea}

\begin{Idea}{Bell States}{}
    \hl{Why Bell states defined in this way}
    Bell states are defined as 
    \begin{align}
        \ket{\Phi^+} &= \frac{1}{\sqrt{2}}(\ket{00}+\ket{11}) \\
        \ket{\Phi^-} &= \frac{1}{\sqrt{2}}(\ket{00}-\ket{11}) \\
        \ket{\Psi^+} &= \frac{1}{\sqrt{2}}(\ket{01}+\ket{10}) \\
        \ket{\Psi^-} &= \frac{1}{\sqrt{2}}(\ket{01}-\ket{10})\,.
    \end{align}
\end{Idea}